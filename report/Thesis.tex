%% ----------------------------------------------------------------
%% Thesis.tex -- MAIN FILE (the one that you compile with LaTeX)
\documentclass[a4paper, 11pt, oneside]{Thesis}  % Use the "Thesis" style, based on the ECS Thesis style by Steve Gunn
\graphicspath{Figures/}  % Location of the graphics files (set up for graphics to be in PDF format)

% Include any extra LaTeX packages required
\usepackage[square, numbers, comma, sort&compress]{natbib}  % Use the "Natbib" style for the references in the Bibliography
\usepackage{verbatim}  % Needed for the "comment" environment to make LaTeX comments
\usepackage{vector}  % Allows "\bvec{}" and "\buvec{}" for "blackboard" style bold vectors in maths
\hypersetup{urlcolor=blue, colorlinks=true}  % Colours hyperlinks in blue, but this can be distracting if there are many links.

%% ----------------------------------------------------------------
\begin{document}
\frontmatter      % Begin Roman style (i, ii, iii, iv...) page numbering

% Set up the Title Page
\title  {Transcription of Percussion Instruments}
\authors  {\texorpdfstring
            {\href{http://sabsathai.github.io/}{Pranav Sankhe}}
            {Pranav Sankhe}
            }
\addresses  {\groupname\\\deptname\\\univname}  % Do not change this here, instead these must be set in the "Thesis.cls" file, please look through it instead
\date       {\today}
\subject    {}
\keywords   {}

\maketitle
%% ----------------------------------------------------------------

\setstretch{1.3}  % It is better to have smaller font and larger line spacing than the other way round

% Define the page headers using the FancyHdr package and set up for one-sided printing

\fancyhead{}  % Clears all page headers and footers
\rhead{\thepage}  % Sets the right side header to show the page number
\lhead{}  % Clears the left side page header

\pagestyle{fancy}  % Finally, use the "fancy" page style to implement the FancyHdr headers

%% ----------------------------------------------------------------
% Declaration Page required for the Thesis, your institution may give you a different text to place here
%\Declaration

%{
%
%\addtocontents{toc}{\vspace{1em}}  % Add a gap in the Contents, for aesthetics
%
%We , AUTHOR NAME, declare that this thesis titled, `THESIS TITLE' and the work presented in it are my own. I confirm that:
%
%\begin{itemize} 
%\item[\tiny{$\blacksquare$}] This work was done wholly or mainly while in candidature for a research degree at this University.
% 
%\item[\tiny{$\blacksquare$}] Where any part of this thesis has previously been submitted for a degree or any other qualification at this University or any other institution, this has been clearly stated.
% 
%\item[\tiny{$\blacksquare$}] Where I have consulted the published work of others, this is always clearly attributed.
% 
%\item[\tiny{$\blacksquare$}] Where I have quoted from the work of others, the source is always given. With the exception of such quotations, this thesis is entirely my own work.
% 
%\item[\tiny{$\blacksquare$}] I have acknowledged all main sources of help.
% 
%\item[\tiny{$\blacksquare$}] Where the thesis is based on work done by myself jointly with others, I have made clear exactly what was done by others and what I have contributed myself.
%\\
%\end{itemize}
% 
% 
%Signed:\\
%\rule[1em]{25em}{0.5pt}  % This prints a line for the signature
% 
%Date:\\
%\rule[1em]{25em}{0.5pt}  % This prints a line to write the date
%}
%\clearpage  % Declaration ended, now start a new page

%% ----------------------------------------------------------------
% The "Funny Quote Page"
%\pagestyle{empty}  % No headers or footers for the following pages
%
%\null\vfill
%% Now comes the "Funny Quote", written in italics
%\textit{``Write a funny quote here.''}
%
%\begin{flushright}
%If the quote is taken from someone, their name goes here
%\end{flushright}
%
%\vfill\vfill\vfill\vfill\vfill\vfill\null
%\clearpage  % Funny Quote page ended, start a new page
%% ----------------------------------------------------------------

% The Abstract Page
\section {Abstract}  % Add the "Abstract" page entry to the Contents

%\addtocontents{toc}{\vspace{1em}}  % Add a gap in the Contents, for aesthetics
The tabla is a percussion instrument of the Indian subcontinent, consisting of a pair of drums, used in traditional, classical, popular and folk music. The audio of tabla is composed of many modalities. We wish to develop an end to end transcription system for tabla audio and to reconstruct the audio from the obtained transcription. We also intend to study the presence of (acoustic-prosodic) correspondences between the recitation and playing of tabla compositions like intensity and F0 variations, at the bol/stroke level (sort of like word-level in speech) and identify those that are well-correlated. In particular for music education, it would be useful to provide a rating and to identify the nature of the error in the learning process.  \\ 
Recently data driven signal processing approaches have been gaining momentum to solve problems like the one elaborated above. The use of machine learning to solve complex non linear problems have been dominating the speech and music research which is evident in the recently published papers. In particular, we have explored the use of NMF (Non-Negative Matrix Factorization) to solve the problem of drum transcription for drum-only-recordings. While this approach works significantly well, having seen the developments in DNN models for Piano transcription, we are motivated to employ a similar architecture for tabla transcription. \\ 
There are few challenges in implementing a neural network architecture for tabla transcription which we will elaborate over in later sections.  We intend to find a way around these challenges and come up with a suitable and robust transcription and evaluation system. \\ 



\clearpage  % Abstract ended, start a new page
%% ----------------------------------------------------------------


\setstretch{1.3}  % Reset the line-spacing to 1.3 for body text (if it has changed)

% The Acknowledgements page, for thanking everyone
%\acknowledgements{
%\addtocontents{toc}{\vspace{1em}}  % Add a gap in the Contents, for aesthetics
%
%The acknowledgements and the people to thank go here, don't forget to include your project advisor\ldots
%
%}
%\clearpage  % End of the Acknowledgements
%% ----------------------------------------------------------------

\pagestyle{fancy}  %The page style headers have been "empty" all this time, now use the "fancy" headers as defined before to bring them back


%% ----------------------------------------------------------------
%\lhead{\emph{Contents}}  % Set the left side page header to "Contents"
\tableofcontents  % Write out the Table of Contents

%% ----------------------------------------------------------------
%\lhead{\emph{List of Figures}}  % Set the left side page header to "List if Figures"
%\listoffigures  % Write out the List of Figures

%% ----------------------------------------------------------------
%\lhead{\emph{List of Tables}}  % Set the left side page header to "List of Tables"
%\listoftables  % Write out the List of Tables

%% ----------------------------------------------------------------
\setstretch{1.5}  % Set the line spacing to 1.5, this makes the following tables easier to read
\clearpage  % Start a new page
%\lhead{\emph{Abbreviations}}  % Set the left side page header to "Abbreviations"
% \listofsymbols{ll}  % Include a list of Abbreviations (a table of two columns)
% {
% % \textbf{Acronym} & \textbf{W}hat (it) \textbf{S}tands \textbf{F}or \\
% \textbf{SL}  & \textbf{S}ign \textbf{L}anguage  \\
% \textbf{NL}  & \textbf{N}atural \textbf{L}anguage\\
% }

%% ----------------------------------------------------------------
%\clearpage  % Start a new page
%\lhead{\emph{Physical Constants}}  % Set the left side page header to "Physical Constants"
%\listofconstants{lrcl}  % Include a list of Physical Constants (a four column table)
%{
%% Constant Name & Symbol & = & Constant Value (with units) \\
%Speed of Light & $c$ & $=$ & $2.997\ 924\ 58\times10^{8}\ \mbox{ms}^{-\mbox{s}}$ (exact)\\
%
%}

%% ----------------------------------------------------------------
%\clearpage  %Start a new page
%\lhead{\emph{Symbols}}  % Set the left side page header to "Symbols"
%\listofnomenclature{lll}  % Include a list of Symbols (a three column table)
%{
%% symbol & name & unit \\
%$a$ & distance & m \\
%$P$ & power & W (Js$^{-1}$) \\
%& & \\ % Gap to separate the Roman symbols from the Greek
%$\omega$ & angular frequency & rads$^{-1}$ \\
%}
%% ----------------------------------------------------------------
% End of the pre-able, contents and lists of things
% Begin the Dedication page

\setstretch{1.3}  % Return the line spacing back to 1.3

%\pagestyle{empty}  % Page style needs to be empty for this page
%\dedicatory{For/Dedicated to/To my\ldots}

%\addtocontents{toc}{\vspace{2em}}  % Add a gap in the Contents, for aesthetics


%% ----------------------------------------------------------------
\mainmatter	  % Begin normal, numeric (1,2,3...) page numbering
\pagestyle{fancy}  % Return the page headers back to the "fancy" style

% Include the chapters of the thesis, as separate files
% Just uncomment the lines as you write the chapters

\chapter{Challenges}

Here we enumerate over the challenges we face in developing a dual objective transcription system for percussion instruments. 

\begin{itemize}
\item Interference of multiple instruments
\item Playing Techniques
\item Recording Conditions and Post Production
\item Insufficient real world datasets
\end{itemize}

\section{Interference of multiple instruments}
The superposition of various instruments makes the recognition of a specific instrument difficult due to the overlaps in both spectral and temporal domain. \\ 
\textbf{Percussive Instruments:} \\ 
A basic drum kit includes drums of different sizes and well- distinguishable timbral characteristics. In a more advanced setup for studio recordings, similar drums with subtle variations in timbre often appear, resulting in sounds that are harder to differentiate. In the case of tabla, we have two indvidual tablas which the player uses to compose music. Seperating and identifying the audio from these two tablas is a research problem in its own right. This problem of interference is more severe when these sounds occur simultaneously. \\ 
\textbf{Melodic Instruments:} \\ 
The wide range of sounds produced from a drum kit or a tabla can potentially coincide with sound components of many melodic instruments. Pre-processing steps for suppression of melodic instruments have been proposed but haven’t been able to achieve substantial improvement. 

\section{Playing Techniques}
Playing techniques is an important aspect of expressive musical performance. For drum instruments, these techniques include basic rudiments as well as timbral variations. 

\textbf{Approaches to model playing techniques:}
\begin{itemize}
\item A study on the automatic identification of timbral variations of the snare drum sounds induced by different excitations has been done where a classification task is formulated to differentiate sounds from different striking locations (center, halfway, edge, etc.) with different excitations (strike, rim shot, and brush)
\item The discrepancy between more expressive gestures on a larger dataset with combinations of different drums, stick heights, stroke intensities, strike positions, and articulations has been explored
\item Different playing techniques for cymbal sounds have been investigated too. The Differentiation is based on either the position where the cymbal is struck (bell, body, edge), how a hi-hat is played (closed, open, chick), or other special effects such as choking a cymbal with the playing hand.
\end{itemize}

All of these studies showed promising results in classifying the isolated sounds, however, when the classifier is applied to the real-world recordings, the performance dropped drastically



\section{Recording Conditions and Post Production}
In practice, it is likely that we have to deal with convolutive, time-variant, and non-linear mixtures instead of linear superpositions of single drum sounds. The acoustic conditions of the recording room and the microphone setup lead to reverberation effects that might be substantial
The recording engineer will likely apply equalization and filtering to the microphone signal
Mostly, the resulting signal alterations can be modeled as convolution with one or more impulse responses.
Non-linear effects such as dynamic compression and distortion might be applied to the drum recordings.

\textbf{Consequences:}
\begin{itemize}
\item Any methods involving machine-learning might deteriorate if the training data does not match the target data.
\item Any methods involving decomposition based on a linear mixture model might be affected when the observed drum mixtures violate these basic assumptions. 
\end{itemize}
A possible strategy to counter these challenges might be data augmentation. In our case the amount of training data could be greatly enhanced by applying diverse combinations of audio processing algorithms including reverberation, distortion and dynamics processing.



\section{Insufficient real world datasets}

\textbf{Size:}
The most common issue of all the existing drum transcription datasets is the insufficient amount of data. Since these datasets are created under very different conditions, they can- not be easily integrated into one large entity. Given that tabla is an Indian classical instrument, the dataset of tabla recordings is limited in size. 

\textbf{Complexity:}
The existing datasets have the tendency of over-simplifying the ADT problem. Only the drum sequences with basic patterns are presented in the dataset.

\textbf{Diversity:}
Most of these datasets do not cover a wide range of music genre and playing style. The limitation in terms of diversity can hinder the system’s capability of analyzing a wider range of music pieces. Particularly, the lack of any singing voice in the corpora ENST-Drums and IDMT-SMT-Drums indicates their insufficiency. Tests on tracks containing singing voice revealed that this poses a big problem, especially for RNN-based ADT methods. 

\textbf{Homogeneity:}
Since each dataset is most likely to be generated under fixed conditions the audio files within the same dataset tend to have high similarities. % Introduction
\chapter{Data Augmentation}

Data Augmentation in the aspect of images have been considerably researched on, as againt for music. There are few papers which we will be following and will also actively ponder on other possible augmentations.  \\ 

\section{Paper 1}
A set of augmentations can be implemented by considering the constrain preserving the labels.
In line with recent research in speech recognition, Schluter Et al., observed that pitch shifting happens to be the most helpful augmentation method. Combined with time stretching and random frequency filtering, they achieved a reduction in classification error between $10$ and $30$, reaching the state of the art on two public datasets. 
Here's the link to their paper in PDF format: \href{http://ofai.at/\~jan.schlueter/pubs/2015_ismir.pdf}{Click Here}. \\ 

\subsection{Data-independent Methods}

A way to increase a model’s robustness is to corrupt training examples with random noise. We consider dropout setting inputs to zero with a given probability and additive Gaussian noise with a given standard deviation. This is fully independent of the kind of data we have, and we apply it directly to the mel spectrograms fed into the network.


\subsection{Audio-specific Methods}
Pitch shifting and time stretching the audio data by moderate amounts does not change the label for a lot of MIR tasks. We implemented this by scaling linear-frequency spectrogram excerpts vertically (for pitch shifting) or horizontally (for time stretching).  \\ 

A much simpler idea focuses on invariance to loudness: We scale linear spectrograms by a random factor in a given decibel range, or, equivalently, add a random offset to log- magnitude mel spectrograms.  \\ 

As a fourth method, we apply random frequency filters to the linear spectrogram. Specifically, we create a filter re- sponse as a Gaussian function.



\section{Paper 2}
 
McFee Et al. have implemeted a software framework in \texttt{Python} for data augmentation for music which remains to be explored in depth but we believe it will provide valuable tools to carry out augmentation efficiently. Here's their paper regarding the same in PDF format: \href{https://bmcfee.github.io/papers/ismir2015_augmentation.pdf}{Click Here} % Introduction
\chapter{NMF for Drum Transcription}

Let $X \epsilon R(K \times T)$ be the magnitude spectrogram where $K$ = frequency bins and $T$ is the time length. We intend to decompose the mixture spectrogram X into spectral basis functions $B(:, r)$ and corresponding time-varying gains $G(r, :)$. Intuitively, speaking, the templates comprise the spectral content of the mixture’s constituent components, while the activations describe when and with which intensity they occur.NMF typically starts with a suitable initialization of matrices B and G. Subsequently, these matrices are iteratively updated to approximate X with respect to a cost function L. The detection of candidate onset events is typically approached by picking the peaks in the activation function G(r,:)
 

\section*{Dataset}
The dataset used was IDMT-SMT Drums. This dataset consists of 608 WAV files (44.1 kHz, Mono, 16bit). The approximate duration is 2:10 hours. Drums used are Kick Drum, Snare Drum and Hi-Hat.. There are 104 polyphonic drum set recordings (drum loops). All the 104 polyphonic drum set recordings are annotated in time and the drum being played. 
The recordings are from three different sources:
\begin{itemize}
\item Real-world, acoustic drum sets (RealDrum) 
\item Drum sample libraries (WaveDrum)
\item Drum synthesizers (TechnoDrum)
\end{itemize}

\section*{Method} 

\subsection*{Obtaining the templates}
\begin{itemize}

\item Get the list of audio and the corresponding annotation files
\item Concatenate all the audio and the corresponding annotations in time
\item Considering the ground truth activations as the initialization of the
activation matrix in NMF, we estimate the basis vectors i.e the
templates for individual drums for each audio file
\item The templates are saved in the file-system as a numpy matrix
\end{itemize}
\subsection*{Predicting Activations}
\begin{itemize}
\item Load the saved templates
\item Compute the spectrogram of the test audio file
\item Considering the generated templates as the initialization of the basis
matrix in NMF, we estimate the activations vectors i.e the activations
for individual drums
\item Apply peak picking on the computed activations in order to select the
prominent bursts
\end{itemize}

\subsection*{Evaluation and Conclusions} 
We created the templates from the drum recordings produced by drum synthesizers and we evaluated our system on the real world drum recordings i.e. drum recordings played by an actual drum-set
\begin{itemize}
\item We used F-measure as our evaluation metric and we use the mir\_eval library to compute the metrics
\item The F-measure we get is 0.659 and the reported F-measure in the literature using NMF on the IDMT-SMT dataset is around 0.7
\item In the current implementation the number of basis vectors assigned is one per drum. We can explore the possibility of assigning more than one basis vectors to a drum.
\item Throughout the evaluation, we can observe that the Snare Drum(SD) in particular have many spurious activations of significant magnitude. Tampering with the computed template of
SD can reduce the spurious activations like smoothening.
\end{itemize}



\begin{figure}[h!]
  \includegraphics[width=6cm, height=6cm]{./images/HH_temp.png}
  \caption{HH template}
  \label{}
\end{figure}




\begin{figure}[h!]
  \includegraphics[width=6cm, height=6cm]{./images/KD_temp.png}
  \caption{KD template}
  \label{}
\end{figure}




\begin{figure}[h!]
  \includegraphics[width=6cm, height=6cm]{./images/SD_temp.png}
  \caption{SD template}
  \label{}
\end{figure}




\begin{figure}[h!]
  \includegraphics[width=6cm, height=6cm]{./images/HH_act.png}
  \caption{HH activation}
  \label{}
\end{figure}




\begin{figure}[h!]
  \includegraphics[width=6cm, height=6cm]{./images/KD_act.png}
  \caption{KD activation}
  \label{}
\end{figure}




\begin{figure}[h!]
  \includegraphics[width=6cm, height=6cm]{./images/SD_act.png}
  \caption{SD activation}
  \label{}
\end{figure}




 % Introduction
\chapter{Literature Review}

 % Introduction
\chapter{Implemented Model}

The model is based on the premise that the frames containing an onset are more important than those who don’t in terms of perception and hence transcription. This idea is manifested by training a dedicated note onset detector and using the raw output of that detector as additional input for the framewise note activation detector. The input data representation used is mel-scaled spectrograms with log amplitude of the input raw audio with 229 logarithmically-spaced frequency bins, a hop length of 512, an FFT window of 2048, and a sample rate of 16kHz.


\section*{Model Description}
The onset detector is composed of the acoustic model, followed by a bidirectional LSTM with 128 units in both the forward and backward directions, followed by a fully connected sigmoid layer with 88 outputs for representing the probability of an onset for each of the 88 piano keys.The frame activation detector is composed of a separate acoustic model, followed by a fully connected sigmoid layer with 88 outputs. Its output is concatenated together with the output of the onset detector and followed by a bidirectional LSTM with 128 units in both the forward and backward directions. Finally, the output of that LSTM is followed by a fully connected sigmoid layer with 88 outputs.
 
\begin{figure}[h!]
  \includegraphics[width=6cm, height=8cm]{./images/piano_model.png}
  \caption{Model Architecture}
  \label{}
\end{figure}

\texttt{Image source: https://arxiv.org/pdf/1710.11153.pdf}


\section*{Dataset}
Datasets of Tabla recordings are very limited in both size and variability. Hence we created our own dataset by taking help from the UPF’s Tabla Solo Dataset. We also used the isolated tabla sounds recorded in our lab itself. \\ 
Strategies to generate tabla recordings:
\begin{itemize}
\item Reading the annotations of the Tabla Solo Dataset and substiuting
the recorded isolated strokes
\item Generate your own compositions with varying tempo value[70, 80, 100]
\item Interchanged beat sequences among the the available scores to
generate new compositions(1415)
\end{itemize}

Training RNNs over long sequences can require large amounts of memory. To expedite training, we split the training audio into smaller files. We split the audio files in 20 second splits which
allowed us to achieve a batch size of 8 during training.



\section*{Model Specifics}
\begin{itemize}
\item The input data representation used is same as that of the magenta’s implementation.
\item The fully connected sigmoid layers had $19$(number of bols).
\item Our loss function is the sum of two cross-entropy losses: one from the onset side and one from the note side.
\item The learning rate was set to $0.0006$ and was decayed with a rate of $0.98$ every $10000$ iterations.
\end{itemize}

\section*{Results}

\begin{figure}[h!]
  \includegraphics[width=6cm, height=6cm]{./images/loss.png}
  \caption{Loss}
  \label{}
\end{figure}

\begin{figure}[h!]
  \includegraphics[width=6cm, height=6cm]{./images/accuracy.png}
  \caption{Accuracy}
  \label{}
\end{figure}


\begin{figure}[h!]
  \includegraphics[width=6cm, height=6cm]{./images/f_measure.png}
  \caption{F-Measure}
  \label{}
\end{figure}


\begin{figure}[h!]
  \includegraphics[width=6cm, height=6cm]{./images/precision.png}
  \caption{precision}
  \label{}
\end{figure}



\begin{figure}[h!]
  \includegraphics[width=6cm, height=6cm]{./images/recall.png}
  \caption{recall}
  \label{}
\end{figure}

\begin{figure}[h!]
  \includegraphics[width=6cm, height=6cm]{./images/false_negatives.png}
  \caption{False Negatives}
  \label{}
\end{figure}

\begin{figure}[h!]
  \includegraphics[width=6cm, height=6cm]{./images/false_positives.png}
  \caption{False Positives}
  \label{}
\end{figure}

\begin{figure}[h!]
  \includegraphics[width=6cm, height=6cm]{./images/true_positives.png}
  \caption{True Positives}
  \label{}
\end{figure}
 

\section*{Discussion}
We shall compare the results of the model which has been implemeted with changes in the architecture and with the state of art system for drums.
\begin{itemize}
\item Both the models indicate to the fact that multitask learning is certainly beneficial. We trained the current model without the onset detector and observed that upto 200 iterations, the f-measure was $15\%$ less $(0.82-0.97)$.
\item Since in the drum transcription model, the beats and the downbeats are computed and used as labels, it renders the results prone to errors. As against here we are joinly detecting the onsets and the bols which do not come from any extra processing but from the dataeset.
\item When you train the model with forward only LSTMs instead of the bi-directional ones, there’s only a small accuracy drop of around
  $8\%$ which is encouraging because forward only LSTMs can be used
\item The results show that near perfect transcription has been achieved.
\item The reason for such high accuracy is that the dataset over which the model has been trained is less complex.
\item Also relatively, transcribing percussion instruments is easy because of the less variability in sound
\item We can possibily achieve better results by combining an acoustic model with a language model
\item Another direction is to go beyond traditional spectrogram representations of audio signals
\item The inference is being carried out after splitting the audio files because the placeholders for data have been hardcoded with the variable shape. This can be undone because TensorFlow allows unspecified variable size
\item Converting the transcriptions to drum recordings to analyse the perception quality of the transcriptions.
\end{itemize}   % Introduction
\newpage


%\input{Chapters/Chapter7} % Conclusion

%% ----------------------------------------------------------------
% Now begin the Appendices, including them as separate files

\addtocontents{toc}{\vspace{2em}} % Add a gap in the Contents, for aesthetics

% \appendix % Cue to tell LaTeX that the following 'chapters' are Appendices

% \chapter{Pre-Training}

\chapter{Sequence to Sequence Model}






	% Appendix Title

%\input{Appendices/AppendixB} % Appendix Title

%\input{Appendices/AppendixC} % Appendix Title

\addtocontents{toc}{\vspace{2em}}  % Add a gap in the Contents, for aesthetics
\backmatter

%% ----------------------------------------------------------------
\label{Bibliography}
\lhead{\emph{Bibliography}}  % Change the left side page header to "Bibliography"
\bibliographystyle{unsrtnat}  % Use the "unsrtnat" BibTeX style for formatting the Bibliography
\bibliography{Bibliography}  % The references (bibliography) information are stored in the file named "Bibliography.bib"

\end{document}  % The End
%% ----------------------------------------------------------------