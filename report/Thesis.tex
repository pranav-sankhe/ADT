%% ----------------------------------------------------------------
%% Thesis.tex -- MAIN FILE (the one that you compile with LaTeX)
\documentclass[a4paper, 11pt, oneside]{Thesis}  % Use the "Thesis" style, based on the ECS Thesis style by Steve Gunn
\graphicspath{Figures/}  % Location of the graphics files (set up for graphics to be in PDF format)

% Include any extra LaTeX packages required
\usepackage[square, numbers, comma, sort&compress]{natbib}  % Use the "Natbib" style for the references in the Bibliography
\usepackage{verbatim}  % Needed for the "comment" environment to make LaTeX comments
\usepackage{vector}  % Allows "\bvec{}" and "\buvec{}" for "blackboard" style bold vectors in maths
\hypersetup{urlcolor=blue, colorlinks=true}  % Colours hyperlinks in blue, but this can be distracting if there are many links.

%% ----------------------------------------------------------------
\begin{document}
\frontmatter      % Begin Roman style (i, ii, iii, iv...) page numbering

% Set up the Title Page
\title  {Tabla transcription using DNN}
\authors  {\texorpdfstring
            {\href{http://sabsathai.github.io/}{Pranav Sankhe}}
            {Pranav Sankhe}
            }
\addresses  {\groupname\\\deptname\\\univname}  % Do not change this here, instead these must be set in the "Thesis.cls" file, please look through it instead
\date       {\today}
\subject    {}
\keywords   {}

\maketitle
%% ----------------------------------------------------------------

\setstretch{1.3}  % It is better to have smaller font and larger line spacing than the other way round

% Define the page headers using the FancyHdr package and set up for one-sided printing

\fancyhead{}  % Clears all page headers and footers
\rhead{\thepage}  % Sets the right side header to show the page number
\lhead{}  % Clears the left side page header

\pagestyle{fancy}  % Finally, use the "fancy" page style to implement the FancyHdr headers

%% ----------------------------------------------------------------
% Declaration Page required for the Thesis, your institution may give you a different text to place here
%\Declaration

%{
%
%\addtocontents{toc}{\vspace{1em}}  % Add a gap in the Contents, for aesthetics
%
%We , AUTHOR NAME, declare that this thesis titled, `THESIS TITLE' and the work presented in it are my own. I confirm that:
%
%\begin{itemize} 
%\item[\tiny{$\blacksquare$}] This work was done wholly or mainly while in candidature for a research degree at this University.
% 
%\item[\tiny{$\blacksquare$}] Where any part of this thesis has previously been submitted for a degree or any other qualification at this University or any other institution, this has been clearly stated.
% 
%\item[\tiny{$\blacksquare$}] Where I have consulted the published work of others, this is always clearly attributed.
% 
%\item[\tiny{$\blacksquare$}] Where I have quoted from the work of others, the source is always given. With the exception of such quotations, this thesis is entirely my own work.
% 
%\item[\tiny{$\blacksquare$}] I have acknowledged all main sources of help.
% 
%\item[\tiny{$\blacksquare$}] Where the thesis is based on work done by myself jointly with others, I have made clear exactly what was done by others and what I have contributed myself.
%\\
%\end{itemize}
% 
% 
%Signed:\\
%\rule[1em]{25em}{0.5pt}  % This prints a line for the signature
% 
%Date:\\
%\rule[1em]{25em}{0.5pt}  % This prints a line to write the date
%}
%\clearpage  % Declaration ended, now start a new page

%% ----------------------------------------------------------------
% The "Funny Quote Page"
%\pagestyle{empty}  % No headers or footers for the following pages
%
%\null\vfill
%% Now comes the "Funny Quote", written in italics
%\textit{``Write a funny quote here.''}
%
%\begin{flushright}
%If the quote is taken from someone, their name goes here
%\end{flushright}
%
%\vfill\vfill\vfill\vfill\vfill\vfill\null
%\clearpage  % Funny Quote page ended, start a new page
%% ----------------------------------------------------------------

% The Abstract Page
\section {Abstract}  % Add the "Abstract" page entry to the Contents

%\addtocontents{toc}{\vspace{1em}}  % Add a gap in the Contents, for aesthetics
The tabla is a percussion instrument of the Indian subcontinent, consisting of a pair of drums, used in traditional, classical, popular and folk music. The audio of tabla is composed of many modalities. We wish to develop an end to end transcription system for tabla audio and to reconstruct the audio from the obtained transcription. We also intend to study the presence of (acoustic-prosodic) correspondences between the recitation and playing of tabla compositions like intensity and F0 variations, at the bol/stroke level (sort of like word-level in speech) and identify those that are well-correlated. In particular for music education, it would be useful to provide a rating and to identify the nature of the error in the learning process.  \\ 
Recently data driven signal processing approaches have been gaining momentum to solve problems like the one elaborated above. The use of machine learning to solve complex non linear problems have been dominating the speech and music research which is evident in the recently published papers. In particular, we have explored the use of NMF (Non-Negative Matrix Factorization) to solve the problem of drum transcription for drum-only-recordings. While this approach works significantly well, having seen the developments in DNN models for Piano transcription, we are motivated to employ a similar architecture for tabla transcription. \\ 
There are few challenges in implementing a neural network architecture for tabla transcription which we will elaborate over in later sections.  We intend to find a way around these challenges and come up with a suitable and robust transcription and evaluation system. \\ 



\clearpage  % Abstract ended, start a new page
%% ----------------------------------------------------------------


\setstretch{1.3}  % Reset the line-spacing to 1.3 for body text (if it has changed)

% The Acknowledgements page, for thanking everyone
%\acknowledgements{
%\addtocontents{toc}{\vspace{1em}}  % Add a gap in the Contents, for aesthetics
%
%The acknowledgements and the people to thank go here, don't forget to include your project advisor\ldots
%
%}
%\clearpage  % End of the Acknowledgements
%% ----------------------------------------------------------------

\pagestyle{fancy}  %The page style headers have been "empty" all this time, now use the "fancy" headers as defined before to bring them back


%% ----------------------------------------------------------------
%\lhead{\emph{Contents}}  % Set the left side page header to "Contents"
\tableofcontents  % Write out the Table of Contents

%% ----------------------------------------------------------------
%\lhead{\emph{List of Figures}}  % Set the left side page header to "List if Figures"
%\listoffigures  % Write out the List of Figures

%% ----------------------------------------------------------------
%\lhead{\emph{List of Tables}}  % Set the left side page header to "List of Tables"
%\listoftables  % Write out the List of Tables

%% ----------------------------------------------------------------
\setstretch{1.5}  % Set the line spacing to 1.5, this makes the following tables easier to read
\clearpage  % Start a new page
%\lhead{\emph{Abbreviations}}  % Set the left side page header to "Abbreviations"
% \listofsymbols{ll}  % Include a list of Abbreviations (a table of two columns)
% {
% % \textbf{Acronym} & \textbf{W}hat (it) \textbf{S}tands \textbf{F}or \\
% \textbf{SL}  & \textbf{S}ign \textbf{L}anguage  \\
% \textbf{NL}  & \textbf{N}atural \textbf{L}anguage\\
% }

%% ----------------------------------------------------------------
%\clearpage  % Start a new page
%\lhead{\emph{Physical Constants}}  % Set the left side page header to "Physical Constants"
%\listofconstants{lrcl}  % Include a list of Physical Constants (a four column table)
%{
%% Constant Name & Symbol & = & Constant Value (with units) \\
%Speed of Light & $c$ & $=$ & $2.997\ 924\ 58\times10^{8}\ \mbox{ms}^{-\mbox{s}}$ (exact)\\
%
%}

%% ----------------------------------------------------------------
%\clearpage  %Start a new page
%\lhead{\emph{Symbols}}  % Set the left side page header to "Symbols"
%\listofnomenclature{lll}  % Include a list of Symbols (a three column table)
%{
%% symbol & name & unit \\
%$a$ & distance & m \\
%$P$ & power & W (Js$^{-1}$) \\
%& & \\ % Gap to separate the Roman symbols from the Greek
%$\omega$ & angular frequency & rads$^{-1}$ \\
%}
%% ----------------------------------------------------------------
% End of the pre-able, contents and lists of things
% Begin the Dedication page

\setstretch{1.3}  % Return the line spacing back to 1.3

%\pagestyle{empty}  % Page style needs to be empty for this page
%\dedicatory{For/Dedicated to/To my\ldots}

%\addtocontents{toc}{\vspace{2em}}  % Add a gap in the Contents, for aesthetics


%% ----------------------------------------------------------------
\mainmatter	  % Begin normal, numeric (1,2,3...) page numbering
\pagestyle{fancy}  % Return the page headers back to the "fancy" style

% Include the chapters of the thesis, as separate files
% Just uncomment the lines as you write the chapters

\chapter{Challenges}

Here we enumerate over the challenges we face in developing a dual objective transcription system for percussion instruments. 

\begin{itemize}
\item Interference of multiple instruments
\item Playing Techniques
\item Recording Conditions and Post Production
\item Insufficient real world datasets
\end{itemize}

\section{Interference of multiple instruments}
The superposition of various instruments makes the recognition of a specific instrument difficult due to the overlaps in both spectral and temporal domain. \\ 
\textbf{Percussive Instruments:} \\ 
A basic drum kit includes drums of different sizes and well- distinguishable timbral characteristics. In a more advanced setup for studio recordings, similar drums with subtle variations in timbre often appear, resulting in sounds that are harder to differentiate. In the case of tabla, we have two indvidual tablas which the player uses to compose music. Seperating and identifying the audio from these two tablas is a research problem in its own right. This problem of interference is more severe when these sounds occur simultaneously. \\ 
\textbf{Melodic Instruments:} \\ 
The wide range of sounds produced from a drum kit or a tabla can potentially coincide with sound components of many melodic instruments. Pre-processing steps for suppression of melodic instruments have been proposed but haven’t been able to achieve substantial improvement. 

\section{Playing Techniques}
Playing techniques is an important aspect of expressive musical performance. For drum instruments, these techniques include basic rudiments as well as timbral variations. 

\textbf{Approaches to model playing techniques:}
\begin{itemize}
\item A study on the automatic identification of timbral variations of the snare drum sounds induced by different excitations has been done where a classification task is formulated to differentiate sounds from different striking locations (center, halfway, edge, etc.) with different excitations (strike, rim shot, and brush)
\item The discrepancy between more expressive gestures on a larger dataset with combinations of different drums, stick heights, stroke intensities, strike positions, and articulations has been explored
\item Different playing techniques for cymbal sounds have been investigated too. The Differentiation is based on either the position where the cymbal is struck (bell, body, edge), how a hi-hat is played (closed, open, chick), or other special effects such as choking a cymbal with the playing hand.
\end{itemize}

All of these studies showed promising results in classifying the isolated sounds, however, when the classifier is applied to the real-world recordings, the performance dropped drastically



\section{Recording Conditions and Post Production}
In practice, it is likely that we have to deal with convolutive, time-variant, and non-linear mixtures instead of linear superpositions of single drum sounds. The acoustic conditions of the recording room and the microphone setup lead to reverberation effects that might be substantial
The recording engineer will likely apply equalization and filtering to the microphone signal
Mostly, the resulting signal alterations can be modeled as convolution with one or more impulse responses.
Non-linear effects such as dynamic compression and distortion might be applied to the drum recordings.

\textbf{Consequences:}
\begin{itemize}
\item Any methods involving machine-learning might deteriorate if the training data does not match the target data.
\item Any methods involving decomposition based on a linear mixture model might be affected when the observed drum mixtures violate these basic assumptions. 
\end{itemize}
A possible strategy to counter these challenges might be data augmentation. In our case the amount of training data could be greatly enhanced by applying diverse combinations of audio processing algorithms including reverberation, distortion and dynamics processing.



\section{Insufficient real world datasets}

\textbf{Size:}
The most common issue of all the existing drum transcription datasets is the insufficient amount of data. Since these datasets are created under very different conditions, they can- not be easily integrated into one large entity. Given that tabla is an Indian classical instrument, the dataset of tabla recordings is limited in size. 

\textbf{Complexity:}
The existing datasets have the tendency of over-simplifying the ADT problem. Only the drum sequences with basic patterns are presented in the dataset.

\textbf{Diversity:}
Most of these datasets do not cover a wide range of music genre and playing style. The limitation in terms of diversity can hinder the system’s capability of analyzing a wider range of music pieces. Particularly, the lack of any singing voice in the corpora ENST-Drums and IDMT-SMT-Drums indicates their insufficiency. Tests on tracks containing singing voice revealed that this poses a big problem, especially for RNN-based ADT methods. 

\textbf{Homogeneity:}
Since each dataset is most likely to be generated under fixed conditions the audio files within the same dataset tend to have high similarities. % Introduction
\chapter{Design patterns a transcription system}

Basic design patterns of a transcription system are enumerated as follows:
\begin{itemize}
\item \textbf{Feature Representation}
\item \textbf{Event Segmentation}
\item \textbf{Activation Function}
\item \textbf{Feature Transformation}
\item \textbf{Event Classification}
\item \textbf{Language Model}
\end{itemize}

\section*{Feature Representation}

Audio signals can be represented into feature representations that are better suited for certain processing tasks. A natural choice is  Short Time Fourier Transform(STFT). These representations are beneficial for untangling and emphasizing the important information hidden in the audio signal. These includes Band-pass filters with predefined center frequencies and bandwidths, Harmonic-Percussive Source Separation, etc.

\section*{Event Segmentation}

Event Segmentation involves detecting the temporal location of musical events in a continuous audio stream. We compute suitable novelty functions and employ various Peak picking strategies to extract local extrema. Recently, learned features using Machine Learing techniques have shown to yield superior performance as compared to hand-crafted ones for event segmentation as often the handcrafted features are approximates. 

\section*{Activation Function}
Map feature representations into activation functions(AF) which indicated the activity level of drum instruments. Techniques which fall under the roof of activation function are NMF, Probabilistic Latent Component Analysis, Deep Neural Networks, etc.
The defining factor of this approach is the concept AF, which generates the activity of a specific instrument over time. With the activation functions for every drum instrument, the event segmentation step can be as simple as finding local maxima of those activation functions by means of suitable peak-picking algorithms. Two families of algorithms for deriving activation functions:
\begin{itemize}
\item \textbf{Matrix Factorization Algorithms}
\item \textbf{Deep Neural Networks}
\end{itemize}

\subsection*{Matrix Factorization Algorithms}

This approach uses magnitude spectrograms and applies matrix factorization algorithms in order to decompose the spectrogram into basis functions and their corresponding activation functions. Early systems used methods such as Independent Subspace Analysis, Prior Subspace Analysis and Non-Negative Independent Component Analysis. The basic assumption of these algorithms is that the target signal is a superposition of multiple, statistically independent sources. This assumption is problematic since the activations of the different drum instruments are usually rhythmically related. When the signal contains both drums and melodic instruments, this assumption may be more severely violated. Recently, more and more systems opted for NMF, which has less strict statistical assumptions about the sources. In NMF, the only constraint is the non-negativity of the sources, which is naturally given in magnitude spectograms. NMF-based ADT systems include basic NMF as well as related concepts such as Non-negative Vector Decomposition, Non-Negative Matrix Deconvolution, Semi-Adaptive NMF, Partially-Fixed NMF, and Probabilistic Latent Component Analysis (PLCA). Most of these factorization-based methods require a set of predefined basis functions as prior knowledge; when this predefined set does not match well with the components in the target signal, the resulting performance may decrease significantly.



\subsection*{Deep Neural Networks}
DNNs are a machine learning architecture that allow to learn non-linear mappings of arbitrary inputs to target outputs based on training data. They are usually constructed as a cascade of layers consisting of learnable, linear weights and simple non-linear functions. The learning of the weight parameters is performed by variants of gradient descent. \\ 
RNNs can in principle also perform sequence modeling, similar to the more classic methods such as HMM. However, the lack of large amounts of training data and the applied training methods, prohibit RNN to perform well. Some of the factorization-based approaches can also be used to reconstruct the magnitude spectrogram of drum sources and serve as source separators. This type of approach takes care of simultaneous events without the need of introducing combined classes during training. However, when the multiple sources overlap in the spectral domain, cross-talk between activation functions will appear and degrade the performance. Furthermore, the use of
magnitude spectrograms neglects the phase, which could potentially strip away critical information. 


\subsection*{Feature Transformation}
This design pattern provides a transformation of the feature representation to a more compact form. This goal can be achieved by different techniques such as feature selection, Principal Component Analysis (PCA), or Linear Discriminant Analysis (LDA).

\subsection*{Event Classification}
This processing step aims at associating the instrument type (e.g., KD, SD, or HH) with the corresponding musical event. In the majority of papers, this is achieved through machine learning methods that can learn to discriminate the target drum instruments based on training examples. Inexpensive alternatives include clustering and cross-correlation.


\subsection*{Language Model}
This pattern takes the sequential relationship between musical events into account. Usually this is achieved using a probabilistic model capable of learning the musical grammar and inferring the structure of musical events. LMs are based on classical methods such as Hidden Markov Models (HMM) or more recent methods such as RNNs.
 % Introduction
\chapter{Datasets}

\textbf{Tabla Solo Dataset:} \\ 
The Tabla Solo Dataset is a transcribed collection of Tabla solo audio recordings spanning compositions from six different Gharanas of Tabla, played by Pt. Arvind Mulgaonkar. The dataset consists of audio and time aligned bol transcriptions.  \\ 



\textbf{Mridangam Stroke Dataset} \\ 
The Mridangam Stroke dataset is a collection of 7162 audio examples of individual strokes of the Mridangam in various tonics. The dataset comprises of 10 different strokes played on Mridangams with 6 different tonic values.  \\ 


\textbf{Mridangam Tani-avarthanam dataset:}
In Carnatic music, Tani-avarthanam is the solo performance by the percussion ensemble following the main piece of the concert.
[No time labels] \\ 



Datasets remain to be explored extensively. 
 % Introduction
\chapter{State of Art Drum Transcription Model}

Richard Vogl Et. al in their paper, Drum Transcription via Joint Beat And Drum Modeling Using Convolution Reccurent Neural Networks have presented the state of art drum transcription system. They use metadata like meters and tempo as they think it helps to transcribe better. The authors claim that the system has the means to utilize information on the rhythmical structure of a song. They evaluate three different architecture designs in particular viz. recurrent, convolutional, and recurrent-convolutional neural networks and they show that learning beats jointly with drums can be beneficial for the task of drum detection. CRNNs should result in a model, in which the convolutional layers focus on acoustic modeling of the events, while the recurrent layers learn temporal structures of the features. 


\section*{About the Metadata}
\begin{itemize}
\item Additional information required by a musician to perform a piece
\item This information comprises meter, over- all tempo, indicators for bar boundaries, indications for local changes in tempo, dynamics, and playing style of the piece. 
\item To obtain some of this information, beat and down-beat detection methods can be utilized. 
\item Beats provide tempo information
\item downbeats add bar boundaries
\item combination of both provides indication for the meter within the bars
\end{itemize}

\section*{Feature Extraction}
  
\begin{itemize}
\item They use log magnitude spectrogram as an input
\item Window size = 46 milliseconds (2048 samples)
\item The frequency bins mel scaled
\item The positive first order differential over time of this spectrogram is calculated and concatenated.
\end{itemize}  

\section*{Neural Network Models}


\subsection*{Convolutional Neural Network}
\begin{itemize}
\item While plain CNNs do not feature any memory, the spectral context allows the CNN to access surround- ing information during training and classification. 
\item How- ever, a context of 25 frames (250ms) is not enough to find repetitive structures in the rhythm patterns. 
\item Therefore, the CNN can only rely on acoustic, i.e., spectral features of the signal. 
\item with advanced training methods like batch normalization, as well as the advantage that CNNs can easily learn pitch invariant kernels, CNNs are well- equipped to learn a task adequate acoustic model. 
\end{itemize}

\subsection*{Convolutional Bidirectional RNN}
\begin{itemize}
\item Convolutional recurrent neural networks (CRNN) repre- sent a combination of CNNs and RNNs. 
\item In this work, the convolutional layers directly process the input features 
\item The recurrent layers are placed after the convolutional layers and are supposed to serve as a means for the network to learn structural patterns 
\end{itemize}

\subsection*{Drum Detection with Oracle Beat Features}
\begin{itemize}
\item In addition to the input features, the annotated beats, represented as the target functions for beats and downbeats, are included as input features. 
\item Using the results of these experiments, it can be investigated if the prior knowledge of beat and downbeat positions is beneficial for drum detection. 
\end{itemize}

\subsection*{Joint Drum and Beat Detection}
\begin{itemize}
\item As input for training, again, only the spectrogram features are used. 
\item Targets for training of the NNs comprise, in this case, drum and beat activation functions.
\item Beats and drums are closely related, because usually drum pat- tern are repetitive on a bar-level (separated by downbeats) and drum onsets often correlate with beats. 
\end{itemize}






\section*{Commonalities between NN architectures considered}
\begin{itemize}
\item All NNs are trained using the same input features
\item The RNN architectures are implemented as bidirectional RNNs (BRNN) 
\item the output layers consist of three or five sigmoid units, representing three drum instruments under observation (drum only) or three drum instruments plus beat and downbeat (drum and beats) 
\item the NNs are all trained using the RMSprop optimization algorithm 
\end{itemize}  


\section*{Joint Detection} 
\begin{itemize}
\item In this work, neural networks for joint beat and drum detection are trained in a multi-task learning fashion. 
\item It is possible to extract drums and beats separately using existing work and combine the results afterwards
\item They show that it is beneficial to train for both tasks together, allowing a joint model to leverage commonalities of the two problems 
\end{itemize}  


\section*{Discussion} 
\begin{itemize}
\item CNN with a large enough spectral context (25 frames in this work) can detect drum events better than RNNs.
\item The results for CNNs and CRNNs show that convolutional feature processing is beneficial for drum detection.
\item The findings considering drum detection results for multi-task learning are also promising
\item The low results of beat and downbeat tracking are a limiting factor
\item As a next step, to better leverage multi-task learning effects, beat detection results must be improved
\item Trained RNNs seem to learn only an acoustic, but not a structural model for drum transcription. (Due to the less number of time-steps used)
\item The difference between the magenta model and the model considered in this paper is that there’s no concatenation of the predicted beats and downbeats. 
\end{itemize}  


 % Introduction
\newpage


%\input{Chapters/Chapter7} % Conclusion

%% ----------------------------------------------------------------
% Now begin the Appendices, including them as separate files

\addtocontents{toc}{\vspace{2em}} % Add a gap in the Contents, for aesthetics

% \appendix % Cue to tell LaTeX that the following 'chapters' are Appendices

% \chapter{Pre-Training}

\chapter{Sequence to Sequence Model}






	% Appendix Title

%\input{Appendices/AppendixB} % Appendix Title

%\input{Appendices/AppendixC} % Appendix Title

\addtocontents{toc}{\vspace{2em}}  % Add a gap in the Contents, for aesthetics
\backmatter

%% ----------------------------------------------------------------
\label{Bibliography}
\lhead{\emph{Bibliography}}  % Change the left side page header to "Bibliography"
\bibliographystyle{unsrtnat}  % Use the "unsrtnat" BibTeX style for formatting the Bibliography
\bibliography{Bibliography}  % The references (bibliography) information are stored in the file named "Bibliography.bib"

\end{document}  % The End
%% ----------------------------------------------------------------