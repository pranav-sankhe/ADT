\chapter{Data Augmentation}

Data Augmentation in the aspect of images have been considerably researched on, as againt for music. There are few papers which we will be following and will also actively ponder on other possible augmentations.  \\ 

\section{Paper 1}
A set of augmentations can be implemented by considering the constrain preserving the labels.
In line with recent research in speech recognition, Schluter Et al., observed that pitch shifting happens to be the most helpful augmentation method. Combined with time stretching and random frequency filtering, they achieved a reduction in classification error between $10$ and $30$, reaching the state of the art on two public datasets. 
Here's the link to their paper in PDF format: \href{http://ofai.at/\~jan.schlueter/pubs/2015_ismir.pdf}{Click Here}. \\ 

\subsection{Data-independent Methods}

A way to increase a model’s robustness is to corrupt training examples with random noise. We consider dropout setting inputs to zero with a given probability and additive Gaussian noise with a given standard deviation. This is fully independent of the kind of data we have, and we apply it directly to the mel spectrograms fed into the network.


\subsection{Audio-specific Methods}
Pitch shifting and time stretching the audio data by moderate amounts does not change the label for a lot of MIR tasks. We implemented this by scaling linear-frequency spectrogram excerpts vertically (for pitch shifting) or horizontally (for time stretching).  \\ 

A much simpler idea focuses on invariance to loudness: We scale linear spectrograms by a random factor in a given decibel range, or, equivalently, add a random offset to log- magnitude mel spectrograms.  \\ 

As a fourth method, we apply random frequency filters to the linear spectrogram. Specifically, we create a filter re- sponse as a Gaussian function.



\section{Paper 2}
 
McFee Et al. have implemeted a software framework in \texttt{Python} for data augmentation for music which remains to be explored in depth but we believe it will provide valuable tools to carry out augmentation efficiently. Here's their paper regarding the same in PDF format: \href{https://bmcfee.github.io/papers/ismir2015_augmentation.pdf}{Click Here}